\chapter{Background}\label{chap:background}

In the previous chapter, we have seen that the CFSTP has been tackled using approaches
from all fields of research reviewed
\cite{ramchurn2010cfstp,ramchurn2010fms,ramchurn2015a,baker2016,nelke2020,tkach2021}. As
shown in Chapter \ref{chap:intro}, and as will also be evident below, the reason is that
it generalises or shares similarities with classic combinatorial optimisation problems.

This chapter will serve as the basis for the rest of the thesis. In Section
\ref{sec:cfstp}, we present an improved version of the constraint program of the CFSTP
given by \cite{ramchurn2010cfstp}. More precisely, we extend the definition of coalition
value, define the constraints with fewer and simpler equations, and introduce the concept
of \emph{solution degree}. In Section \ref{sec:cfla}, we illustrate CFLA, the
state-of-the-art CFSTP algorithm. In Chapter \ref{chap:contrib1}, we improve the design of
CFLA, then we eliminate its limitations by formulating a novel algorithm. Finally, we give
the original \emph{Mixed Integer Program} (MIP) \cite{wolsey2020} of the CFSTP in Section
\ref{sec:mip}, which will be shown to be equivalent to a novel and significantly shorter
program in Chapter \ref{chap:contrib2}.

\section{The CFSTP Model}\label{sec:cfstp}

We first give our basic definitions (Section \ref{sec:defs}), then characterise coalition
allocation and coalition values (Sections \ref{sec:ca} $-$ \ref{sec:cv}), and finally give
the constraints (Section \ref{sec:constraints}) and the objective function of the CFSTP
(Section \ref{sec:of}).

\subsection{Basic Definitions}\label{sec:defs}

Let $V = \{ v_1, \dots, v_m \}$ be a set of $m$ tasks and $A = \{ a_1, \dots, a_n \}$ be a
set of $n$ agents. Let $L$ be the finite set of all possible task and agent locations.
Time is denoted by $t \in \mathbb{N}$, starting at $t = 0$, and agents travel or complete
tasks with a base time unit of $1$. The time units needed by an agent to travel from one
location to another are given by the function $\rho : A \times L \times L \rightarrow
\mathbb{N}$. Having $A$ in the domain of $\rho$ allows to characterise different agent
features (e.g., speed or type). Let $l_v$ be the fixed location of task $v$, and let
$l_a^t \in L$ be the location of agent $a$ at time $t$, where $l_a^0$ is its initial
location and is known a priori.
Each task $v$ has a \emph{demand} $(\gamma_v, w_v)$ such that $\gamma_v$ is the
\emph{deadline} of $v$, or the time until which agents can work on $v$
\cite{nunes2017taxonomy}, and $w_v \in \mathbb{R}_{\geq 0}$ is the \emph{workload} of $v$,
or the amount of work required to complete $v$.
We denote the location of agent $a$ at time $t$ by $l_a^t \in L$, the
times at which $a$ starts and finishes working on task $v$ by $s_a^v \in
[0, \gamma_v]$ and $f_a^v \in [s_a^v, \gamma_v]$, respectively, and the \emph{maximum
problem time} by $\tmax = \max_{v \in V} \gamma_v$.

\subsection{Coalition Allocations}\label{sec:ca}

A subset of agents $C \subseteq A$ is called a \emph{coalition}. At time $t$, the
rationale for allocating coalition $C$ to task $v$ is that $C$ completes $v$ in the fewest
time units. An \emph{agent allocation} is denoted by $\alloc{a}{t}$ and represents the
fact that agent $a$ works on task $v$ at time $t$. The \emph{set of all agent allocations}
is denoted by:
\begin{equation}
    T = \left\{ \alloc{a}{t} \right\}_{a \in A,\, v \in V,\, t \in [0,\, \tmax]}
\end{equation}
and contains all possible agent allocations.
A \emph{coalition allocation} is denoted by $\alloc{C}{t}$ and represents the fact that
coalition $C$ works on task $v$ at time $t$. Given a set of agent allocations $T'
\subseteq T$, and a time $t' \leq \tmax$, the \emph{set of coalition allocations
corresponding to} $T'$ over the time period $[0, t']$ is denoted by:
\begin{equation}\label{eq:gtp}
    \Gamma(T', t') = \left\{ \tau_t^{C \rightarrow v} \, | \, C = \left\{ a\; | \;
    \tau_t^{a \rightarrow v} \in T' \right\}, t \leq t' \right\}
\end{equation}
Furthermore, the \emph{set of all coalition allocations} is denoted by:
\begin{equation}
    \Gamma = \Gamma(T, \tmax)
\end{equation}
Similar to $T$, $\Gamma$ contains all possible coalition allocations.
An agent allocation $\alloc{a}{t}$ is also denoted as a \emph{singleton coalition
allocation} $\alloc{\{a\}}{t}$.

\subsection{Coalition Values}\label{sec:cv}

A subset of agents $C \subseteq A$ is called a \emph{coalition}. Each coalition
allocation $\alloc{C}{t}$ has a \emph{coalition value}, given by the function\footnote{In
cooperative game theory, this is called a \emph{characteristic function} \cite[Section
$2.1$]{chalkiadakis2011}.} $u : P(A) \times V \rightarrow \mathbb{R}_{\geq 0}$, where
$P(A)$ is the power set of $A$.
The value of $u(C, v)$ expresses how well the agents in $C$ work together on $v$, and the
workload $w_v$ decreases by $u(C, v)$ at each time.
\iffalse
Coalitions need not to be super-additive. Tasks are heterogeneous, in the sense that
they may have different workloads and deadlines.
\fi

\subsection{Constraints}\label{sec:constraints}

There are $3$ types of constraints: structural, temporal and spatial. Structural
constraints require that each task $v$ can be allocated to only one coalition at a time.
This is characterised by the following sets:
\begin{equation}\label{eq:c0}
    \forall v \in V, \Gamma_v = \left\{ \Gamma' \subseteq \Gamma :
    \alloc{C_1}{t}, \alloc{C_2}{t} \in \Gamma' \Rightarrow C_1 = C_2 \right\}
\end{equation}
With an abuse of notation, we write $\alloc{C}{t} \in \Gamma_v$
to indicate that $\alloc{C}{t}$ belongs to an unspecified set of $\Gamma_v$.
\iffalse
In other words, if $\alloc{a_1}{t}$, $\alloc{a_2}{t} \in T$, then $\alloc{ \{ a_1, a_2
\}}{t} \in \Gamma_v$.
\fi
Temporal constraints require that each task $v$ can be completed only by its deadline
$\gamma_v$. This is characterised by the function $\Delta : V \times \Gamma \rightarrow
\{0, 1 \}$, defined as follows:
\begin{equation}\label{eq:t1}
    \Delta(v, \Gamma) =
    \begin{cases}
        1 \text{,} & \quad \text{if }\, \exists\, t \leq \gamma_v :
        \sum_{t' \leq t \text{, } \alloc{C}{t'} \in \Gamma_v} u(C, v) \geq w_v\\
        0 \text{,} & \quad \text{otherwise}
    \end{cases}
\end{equation}
Equation \ref{eq:t1} utilises $\Gamma_v$ (Equation \ref{eq:c0}) to count only coalition
allocations that satisfy the structural constraints.
Spatial constraints require that an agent will not start working on a task
before reaching it. This is characterised as follows:
\begin{gather}
    \forall a \in A,\, \forall v \in V, \forall t \leq \gamma_v,\; s_a^v \geq t +
    \rho(a, l_a^t, l_v)\label{eq:s1}\\
    \forall a \in A, \forall v_1, v_2 \in V,\; f_a^{v_1} + \rho(a, l_{v_1},
    l_{v_2}) \leq s_a^{v_2}\label{eq:s2}
\end{gather}
A set of agent allocations $T' \subseteq T$ is called \emph{legal} if it exists
a time $t' \leq \tmax$ such that $\Gamma(T', t')$ satisfies Equation \ref{eq:t1}.
A set of coalition allocations $\Gamma' \subseteq \Gamma$ that satisfies
Equations \ref{eq:t1} $-$ \ref{eq:s2} is called \emph{feasible}. Consequently, at time
$t$, if $\tau_{t}^{C_1 \rightarrow v_1}$ and $\tau_{t}^{C_2 \rightarrow v_2}$ are feasible
coalition allocations and $l_{v_1} \neq l_{v_2}$, then $C_1 \cap C_2 = \emptyset$.

There are no synchronisation constraints \cite{nunes2017taxonomy}: when a task $v$
is allocated to a coalition $C$, each agent $a \in C$ starts working on $v$ as soon as it
reaches $l_v$. % without waiting for the remaining agents.
Hence, $v$ is completed by a temporal sequence of subcoalitions of $C$.

\subsection{Objective Function}\label{sec:of}

The objective function of the CFSTP is to find a set of feasible coalition
allocations that maximises the number of completed tasks:
\begin{equation}
    \arg \max_{\substack{\Gamma' \subseteq \Gamma\\\Gamma'
    \text{feasible}}}\; \sum_{v \in V} \Delta(v, \Gamma') \label{eq:of}
\end{equation}
To solve Equation \ref{eq:of}, an exhaustive search may require to verify all the possible
coalition allocations until $\tmax$. Consequently, the time complexity of
searching for an optimal solution is:
\begin{equation}\label{eq:cfstpcomp}
    O \left( |V|! \cdot 2^{|A|} \cdot \tmax \right)
\end{equation}
A set of feasible coalition allocations $\Gamma' \subseteq \Gamma$ is called a
\emph{solution with degree $k$} if $\sum_{v \in V} \Delta(v, \Gamma') = k$.
Hence, an argument of Equation \ref{eq:of} is a solution with the highest degree. Since
it generalises the TOP (Figure \ref{fig:relationships}), the CFSTP is NP-hard
\cite{papadimitriou1993}.

\section{The CFLA Algorithm}\label{sec:cfla}

In this section, we report the concept of CFLA and the procedures of which it is composed.
CFLA has $4$ phases, but \cite[Section $6$]{ramchurn2010cfstp} describes them in $3$
procedures. For readability purposes, we describe them in $4$.

\subsection{The Concept of CFLA}\label{sec:cfla_gen}

CFLA is a centralised, anytime and greedy algorithm that solves Equation \ref{eq:of} by
maximising the working time of agents and minimising the time required by coalitions
to complete tasks. It is divided into $4$ phases:
\begin{enumerate}
    \item Defining the legal agent allocations (Section \ref{sec:cfla1});
    \item For each task $v$, choosing the best coalition $C$ (Section \ref{sec:cfla2});
    \item For each task $v$, doing a $1$-step look-ahead (Section \ref{sec:cfla3}) to
        define its \emph{degree} $\delta_v$, or the number of tasks that can be completed
        after the completion of $v$;
    \item At each time $t \leq \tmax$, allocating a task not yet completed and with the
        highest degree (Section \ref{sec:cfla4}).
\end{enumerate}

\subsection{Phase 1: Defining the Legal Agent Allocations}\label{sec:cfla1}

At time $t$, Algorithm \ref{algo:alloc} determines which free agents\footnote{That is,
agents who neither are travelling to nor working on a task.} $A^t_{free}$ can reach
which uncompleted tasks $V_{unc}$ by their deadlines. The resulting set of legal
agent allocations is denoted by $\text{Leg}_t$.

\begin{algorithm}[t]
  \DontPrintSemicolon
  \KwIn{time $t$}
  $\text{Leg}_t \gets \emptyset$ \Comment*[h]{\small the set of legal agent allocations at
  time $t$}\;
  \For(\Comment*[h]{\small for each free agent $a$}){$a \in A^t_{free}$}{
      \For(\Comment*[h]{\small for each uncompleted task $v$}){$v \in V_{unc}$}{
          \If(\Comment*[h]{\small if $a$ can reach $v$ at $t$ by $\gamma_v$}){$t +
              \rho(a, l_a^t, l_v) \leq \gamma_v$}{
              $\text{Leg}_t \gets \text{Leg}_t \cup \{ \alloc{a}{t'} \}_{t + \rho(a,
              l_a^t, l_v) \leq t' \leq \gamma_v}$\;
          }
      }
  }
  \Return $\text{Leg}_t$\;
  \caption{\textsf{getLegalAgentAllocations} (Phase $1$ of CFLA)\label{algo:alloc}}
\end{algorithm}

\subsection{Phase 2: Selecting the Best Coalition for Each Task}\label{sec:cfla2}

\begin{algorithm}[t]
  \DontPrintSemicolon
  \KwIn{task $v$, a set of legal agent allocations $\text{Leg}_t$}
  $C^*_v \gets \emptyset$ \Comment*[h]{\small an ECF coalition}\;
  Using $\text{Leg}_t$, define $\Gamma(T', \gamma_v)$ as in Equation \ref{eq:gtp}\;
  \tcp*[h]{\small minimise $|C|$, where $C$ can complete $v$ by its deadline $\gamma_v$}\;
  Find $minsize = \min_{\tau_t^{C \to v} \in \Gamma(T', \gamma_v)} |C|$ where
  $\sum_{\tau_{t'}^{C \to v} \text{, } t \leq t' \leq \gamma_v} u(C, v) \geq w_v$\;
  $t^{\min}_v \gets \gamma_v$\;
  \tcp*[h]{\small loop through all possible coalitions of size $minsize$}\;
  \For{$\tau_t^{C \to v} \in \Gamma(T', \gamma_v)$ \textnormal{where} $|C| = minsize$}{
      \If(\Comment*[h]{\small $C$ can complete $v$ by its deadline $\gamma_v$}){$\sum_{\tau_{t'}^{C \to v} \text{, } t \leq t' \leq \gamma_v} u(C, v) \geq 0$}{
          \tcp*[h]{\small minimum time at which $C$ can complete $v$}\;
          $t_{\textnormal{minmax}} \gets \min_{t_{\max}} \left( w_v - \sum_{\tau_{t'}^{C \to v} \text{, } t \leq t' \leq
          t_{\max}} u(C, v) \right)$\;
          \If(\Comment*[h]{\small check if $C$ is the new best
              coalition}){$t_{\textnormal{minmax}} < t_v^{\min}$}{
              $t^{\min}_v = t_{\textnormal{minmax}}$\;
          $C^*_v \gets C$\;
          }
      }
  }
  \Return $C^*_v$\;
  \caption{\textsf{ECF} (Phase $2$ of CFLA)\label{algo:ecf}}
\end{algorithm}

Given a task $v$ and a set of legal agent allocations $\text{Leg}_t$ (computed by Algorithm
\ref{algo:alloc}), Algorithm \ref{algo:ecf} returns an \emph{Earliest Completion First}
(ECF) coalition $C^*_v$ that can be allocated to $v$. In other words, $C^*_v$ is chosen
such that the completion time of $v$ is minimised and the agent in $C^*_v$ become free to
work on the remaining uncompleted tasks as soon as possible. This algorithm is myopic
\cite[Section $6.1$]{ramchurn2010cfstp}, since it does not consider the system status
after that $C^*_v$ would complete $v$. For this reason, Ramchurn et al. introduced the
concept of task degree (Section \ref{sec:cfla_gen}) and developed the look-ahead technique
presented below.

\subsection{Phase 3: Defining the Degree of Each Task}\label{sec:cfla3}

\begin{algorithm}[t]
  \DontPrintSemicolon
  \KwIn{task $v$, an ECF coalition $C^*_v$ of $v$, current solution $\Gamma'$}
  $\delta_v \gets 0$ \Comment*[h]{\small the degree of $v$}\;
  $f_v \gets $ time at which $C^*_v$ completes $v$\;
  \For{$v_2 \in V_{unc} \setminus \{ v \}$}{
      $A_{free}^{f_v} \gets$ agents that are free at $f_v$ \Comment*[h]{\small
      derived from $C^*_v$ and $\Gamma'$}\;
      $A^{d_{v_2}} \gets$ select from $A_{free}^{f_v}$ all agents that can reach $v_2$
      by $d_{v_2}$\;
      $i \gets 1$\;
      \While{$i \leq |A^{d_{v_2}}|$}{
          \For{$C \in$ \textnormal{all combinations of $i$ agents in} $A^{d_{v_2}}$}{
              \If(\Comment*[h]{\small if $C$ can complete $v_2$}){$\sum_{\alloc{C'}{t} \in
              \Gamma_v \text{, } C' \subseteq C \text{, } t \in \left[ f_v, d_{v_2}
              \right]} u(C, v) \geq w_v$}{
                  $\delta_v \gets \delta_v + 1$\;
                  $i \gets |A^{d_{v_2}}|$ \Comment*[h]{\small break external loop too}\;
                  \Break
              }
          }
          $i \gets i + 1$\;
      }
  }
  \Return $\delta_v$\;
  \caption{\textsf{lookAhead} (Phase $3$ of CFLA)\label{algo:la}}
\end{algorithm}

Given a task $v$, Algorithm \ref{algo:la} performs a brute-force search to define its
degree $\delta_v$ (Section \ref{sec:cfla_gen}). At line $8$, with a procedure similar to
line $5$ in Algorithm \ref{algo:ecf}, it checks how many tasks can be completed after the
completion of $v$.
Hence, Algorithm \ref{algo:la} assigns a score to each coalition allocation selected by
Phase $2$ (Section \ref{sec:cfla2}) for each currently uncompleted task. These scores are
then used by Phase $4$ (Section \ref{sec:cfla4}) to choose the next task to execute.

\subsection{Phase 4: Overall Procedure of CFLA}\label{sec:cfla4}

Algorithm \ref{algo:cfla} shows the overall procedure. The \textsf{repeat-until} loop runs
until all tasks are completed, or until the maximum problem time is expired (line $22$). At
each time $t$, the set of legal agent allocations is updated (line $8$), and a task
allocation is defined (Lines $9 - 18$). If it is not possible to allocate other tasks,
the algorithm stops early (line $19$).

\begin{algorithm}[t]
  \DontPrintSemicolon
  \KwIn{tasks $V$, task demands $(\gamma_v, w_v)_{v \in V}$, agents $A$, locations $L$,
  travel function $\rho$, coalition value function $u$}
  $t \gets 0$\;
  $\Gamma' \gets \varnothing$
  \Comment*[h]{\small a solution}\;
  $V_{unc} \gets V$ \Comment*[h]{\small uncompleted tasks}\;
  \Repeat{$V_{unc} = \emptyset$ or $t > \tmax$}{
      $\delta_{max} \gets 0$ \Comment*[h]{\small maximum task degree}\;
      $v^* \gets$ \textsc{nil} \Comment*[h]{\small next task to allocate}\;
      $C^* \gets \emptyset$ \Comment*[h]{\small coalition to which $v^*$
      is allocated}\;
      $\text{Leg}_t \gets$ \textsf{getLegalAgentAllocations}$(t)$ \Comment*[h]{\small
      Algorithm \ref{algo:alloc}}\;
      \For{$v \in V_{unc}$}{
          $C^*_v \gets$ \textsf{ECF}$(v, \text{Leg}_t)$ \Comment*[h]{\small
          Algorithm \ref{algo:ecf}}\;
          $\delta_v \gets$ \textsf{lookAhead}$(v, C^*_v, \Gamma')$ \Comment*[h]{\small
          Algorithm \ref{algo:la}}\;
          \If{$\delta_v > \delta_{max}$}{
              $\delta_{max} \gets \delta_v$\;
              $v^* \gets v$\;
              $C^* \gets C^*_v$\;
          }
      }
      \If{$v^* \neq$ \textnormal{\textsc{nil} and} $C^* \neq \emptyset$}{
          Add to $\Gamma'$ the allocation of $C^*$ to $v^*$\;
          $V_{unc} \gets V_{unc} \setminus \{ v^* \}$\;
      }
      \If(\Comment*[h]{\small all agents are free}){$A^t_{free} = A$}{
          \Break\;
      }
      $t \gets t + 1$\;
  }
  \Return $\Gamma'$\;
  \caption{Overall procedure (Phase $4$ of CFLA)\label{algo:cfla}}
\end{algorithm}

\section{The Original Mixed Integer Program of the CFSTP}\label{sec:mip}

As opposed to \cite[Section $5$]{ramchurn2010cfstp}, for readability purposes, we begin by
introducing the decision variables, then we formulate and explain the constraints,
%using typo-free equations and a more concise explanation style,
and finally give the objective function.
To be consistent with the notation introduced so far, we rename some variables and
constraints.

\subsection{Decision Variables}\label{sec:original-dv}
Let $L_V \subseteq L$ denote the set of all task locations.
There are $4$ sets of binary variables and $1$ set of integer variables:
\begin{gather}
    \forall v \in V,\, \forall t \leq \tmax,\, \forall a \in A,\;
    \mxa \in \left\{ 0, 1 \right\}\\
    \forall v \in V,\, \forall t \leq \tmax,\, \forall C \subseteq A,\;
    \mxc \in \left\{ 0, 1 \right\}\label{eq:dvc}\\
    \forall v \in V,\, y_v \in \left\{ 0, 1 \right\}\\
    \forall a \in A,\, \forall l_1 \in L,\, \forall l_2 \in L_V,\; r^a_{l_1, l_2} \in \left\{ 0, 1 \right\}\\
    \forall v \in V,\, \forall a \in A,\;
    \lambda^v_a \in \left\{ 0, \dots, \tmax \right\}
\end{gather}
where: $\mxa = 1$ (resp. $\mxc = 1$) if agent $a$ (resp. coalition $C$) works on task $v$
at time $t$, and $0$ otherwise; $y_v = 1$ if task $v$ is completed, and $0$ otherwise;
$r^a_{l_1, l_2} = 1$ if agent $a$ travels from location $l_1 \in L$ to location $l_2 \in
L_V$, and $0$ otherwise, and $\lambda^v_a$ is the time at which agent $a$ starts working
on task $v$.
It is assumed that the allocation process starts at $t = 1$. Hence, $\lambda^v_a = 0$
indicates that agent $a$ does not work on task $v$.
In the original formulation, $y_v$ is
$\delta_v$, $\mxc$ is $\alloc{C}{t}$, and $\lambda^v_a$ is $s^v_a$.

\subsection{Constraints}

There are $4$ types of constraints: \emph{completion}, \emph{temporal}, \emph{spatial},
and \emph{linking}. In the original formulation, the temporal constraints are called
\emph{deadline} constraints, while the spatial constraints are called \emph{starting time,
routing and service consistency} constraints.

The following equations use the Big-M method \cite{griva2009}: \ref{eq:mip1},
\ref{eq:mip4}, \ref{eq:mip5}, \ref{eq:mip9}, \ref{eq:mip13}, \ref{eq:mip14},
\ref{eq:mip17}, \ref{eq:mip18}, and \ref{eq:mip20}.
This method introduces an arbitrarily large positive constant $M$ to artificially penalise
the violation of the constraints involved.

\paragraph{Completion Constraints}
The work done for each task $v$ is at least equal to its workload $w_v$ if the task is
completed, and $0$ otherwise:
\begin{gather}
    \forall v \in V,\; \sum_{t \leq \tmax} \sum_{C \subseteq A} \mxc \cdot u(C, v) \geq
    y_v \cdot w_v\label{eq:mip0}\\
    \forall v \in V,\; \sum_{t \leq \tmax} \sum_{C \subseteq A} \mxc \leq M \cdot y_v\label{eq:mip1}
\end{gather}
Moreover, at each time, at most one coalition can work on each task:
\begin{equation}\label{eq:mip2}
    \forall v \in V,\, \forall t \leq \tmax,\; \sum_{C \in A} \mxc \leq y_v
\end{equation}

\paragraph{Temporal Constraints}
Each task can only be completed by its deadline:
\begin{equation}\label{eq:mip3}
    \forall v \in V,\, \forall a \in A,\; \lambda^v_a + \sum_{t \leq \tmax} \mxa \leq \gamma_v
\end{equation}
Since it is assumed that there are no allocations at time $t = 0$, it follows from
Equations \ref{eq:mip0} and \ref{eq:mip1} that if a task is not completed, the left-hand
side of Equation \ref{eq:mip3} is $0$.

\paragraph{Spatial Constraints}
For each agent $a$ and task $v$, $a$ can start to work on $v$ after finishing work on a
previous task, or after reaching the task location $l_v$ from its initial location
$l^a_0$:
\begin{gather}
    \forall a \in A,\, \forall l \in L_V \cup \left\{ l_0^a \right\},\, \forall v \in V,\;
    \rho \left(a, l, l_v \right) \leq \lambda^v_a + M \cdot \left( 1 - r^a_{l_a, l_v}
    \right)\label{eq:mip4}\\
    \forall a \in A,\, \forall v_1, v_2 \in V,\;
    \lambda^{v_1}_a + \sum_{t \leq \tmax} x_{v_1,\, t,\, a} + \rho\left(a, l_{v_1},
    l_{v_2}\right) \leq \lambda^{v_2}_a + M \cdot \left( 1 - r^a_{l_{v_1}, l_{v_2}}
    \right)\label{eq:mip5}
\end{gather}
For each agent $a$ and time $t$, $a$ can work on at most one task during $t$:
\begin{equation}\label{eq:mip6}
    \forall a \in A,\, \forall t \leq \tmax,\; \sum_{v \in V} \mxa \leq 1
\end{equation}
If an agent starts working on a task at time $t$, then variables $\mxa$ and $x_{v,\, t -
1,\, a}$ must have different values:
\begin{equation}\label{eq:mip7}
    \forall a \in A,\, \forall v \in V,\, \forall t \in [1, \tmax],\;
    1 - 2 \cdot \left| t - \lambda^v_a \right| \leq \mxa - x_{v,\, t - 1,\, a}
\end{equation}
For each agent $a$ and task $v$, $a$ changes the status of its service on $v$ (e.g.,
\emph{not working} to \emph{working} or vice versa) at most twice:
\begin{gather}
    \forall a \in A,\, \forall v \in V,\;
    \sum_{t \in [1, \tmax]} \left| \mxa - x_{v,\, t - 1,\, a} \right| \leq
    2\label{eq:mip8}\\
    \begin{gathered}
    \forall a \in A,\, \forall v_1, v_2 \in V,\, \forall t \in [1, \tmax],\\ 1 - \left| t
    - \lambda^{v_2}_a - \rho \left( a, l_{v_1}, l_{v_2} \right) \right| - M \cdot \left( 1
    - r^a_{l_{v_1}, l_{v_2}} \right) \leq \left| x_{v_1,\, t,\, a} - x_{v_1,\, t - 1,\, a}
    \right|
    \end{gathered}\label{eq:mip9}
\end{gather}
An agent cannot leave and reach the same task location, it can only reach a new task
location from exactly one previous location, and it can only leave a location to
reach exactly one task location:
\begin{gather}
    \forall a \in A,\, l \in L_V,\; r^a_{l, l} = 0\label{eq:mip10}\\
    \forall a \in A,\, \forall l \in L_V \cup \left\{ l^a_0 \right\},\, \forall v_1 \in
    V,\; r^a_{l, l_{v_1}} + \sum_{v_2 \in V \setminus \{ v_1 \}} r^a_{l_{v_1}, l_{v_2}}
    \leq 1\label{eq:mip11}
\end{gather}

\begin{gather}
    \forall a \in A,\, \forall l \in L_V \cup \left\{ l^a_0 \right\},\;
    \sum_{v \in V} r^a_{l, l_v} \leq 1\label{eq:mip12}\\
    \forall v_2 \in V,\, \forall a \in A,\;
    \sum_{v_1 \in V \setminus \{ v_2 \}} r^a_{l_{v_1}, l_{v_2}} \leq M \cdot \sum_{t \leq
    \tmax} x_{v_2,\, t,\, a}\label{eq:mip13}\\
    \forall v_1 \in V,\, \forall a \in A,\;
    \sum_{v_2 \in V \setminus \{ v_1 \}} r^a_{l_{v_1}, l_{v_2}} \leq M \cdot \sum_{t \leq
    \tmax} x_{v_1,\, t,\, a}\label{eq:mip14}
\end{gather}

\paragraph{Linking Constraints}
An agent reaches a task only to complete it:
\begin{equation}\label{eq:mip15}
    \forall v \in V,\;
    \sum_{a \in A, l \in L_V \cup \left\{ l^a_0 \right\}} r^a_{l, l_v} \geq y_v
\end{equation}
A coalition $C$ works on task $v$ and at time $t$ only if every agent $a \in C$ works on
$v$ at $t$:
\begin{equation}\label{eq:mip16}
    \forall C \subseteq A,\, \forall v \in V,\, \forall t \leq \tmax,\;
    \sum_{a \in C} \mxa \geq |C| \cdot \mxc
\end{equation}
If task $v$ is allocated to agent $a$, then $a$ works on $v$ for at least $1$ unit of
time:
\begin{gather}
    \forall a \in A,\, \forall v \in V,\;
    \lambda^v_a \leq M \cdot \sum_{t \leq \tmax} \mxa\label{eq:mip17}\\
    \forall a \in A,\, \forall v \in V,\;
    \sum_{t \leq \tmax} \mxa \leq M \cdot \lambda^v_a\label{eq:mip18}
\end{gather}
If task $v$ is allocated to agent $a$, then $a$ works on $v$ after reaching $l_v$ from
another task location or from its initial location $l^a_0$:
\begin{gather}
    \forall a \in A,\, \forall v \in V,\;
    \lambda^v_a \geq \sum_{l \in L_V \cup \left\{ l^a_0 \right\}} r^a_{l,
    l_v}\label{eq:mip19}\\
    \forall a \in A,\, \forall v \in V,\;
    \lambda^v_a \leq M \cdot \sum_{l \in L_V \cup \left\{ l^a_0 \right\}} r^a_{l,
    l_v}\label{eq:mip20}
\end{gather}
Each agent can work on at most one task at a time, and cannot work at time $t = 0$:
\begin{gather}
    \forall a \in A,\, \forall t \leq \tmax,\; \sum_{v \in V} \mxa \leq 1\label{eq:mip21}\\
    \forall a \in A,\, \forall v \in V,\; x_{v,\, 0,\, a} = 0\label{eq:mip22}
\end{gather}
Equations \ref{eq:mip8}, \ref{eq:mip16} and \ref{eq:mip22} also imply that if a coalition
$C$ works on a task, the service status of each $a \in C$ switches from $0$ to $1$ at the
beginning of the work, and from $1$ to $0$ at the end of it.

\subsection{Objective Function}\label{sec:of0}

The objective of the CFSTP is to maximise the number of tasks completed:

\begin{equation}\label{eq:cfstpof}
    \max \sum_{v \in V} y_v \text{ subject to Equations } \ref{eq:mip0} - \ref{eq:mip22}
\end{equation}

To solve the MIP defined by Equation \ref{eq:cfstpof}, we first need to create all
decision variables and constraints. Creating the decision variables of Equation
\ref{eq:dvc} requires to list all $\mathcal{L}$-tuples over $P(A)$, where $\mathcal{L} =
|V| \cdot \tmax$, with a worst-case time and space complexity of:
\begin{equation}\label{eq:ccc}
    O\left( {\left(2^{|A|}\right)}^{\mathcal{L}} \right) =
    O\left( 2^{|A| \cdot |V| \cdot \tmax} \right)
\end{equation}
Creating the remaining decision variables and constraints does not take more time and
space than Equation \ref{eq:ccc}.
Hence, implementing and solving the above MIP with optimisation software packages such as
CPLEX or GLPK may require an exponential amount of time and space, followed in the worst
case by a factorial amount of time (Equation \ref{eq:cfstpcomp}).
