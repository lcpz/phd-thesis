\chapter{Large-scale, Dynamic and Distributed CFSTP}\label{chap:contrib2}

After identifying the principal shortcomings in the CFSTP literature in Section
\ref{sec:cfstp-gaps}, this chapter proposes a minimal mathematical formulation in Section
\ref{sec:cfstp-bip}, a distributed version of CTS in Section \ref{sec:dcts}, and the first
large-scale, dynamic and distributed CFSTP test framework in Section \ref{sec:dcts-tests}.

\section{Major Gaps in the CFSTP Literature}\label{sec:cfstp-gaps}

There are $3$ main issues in the CFSTP literature. First, its original mathematical
programming formulation (Section \ref{sec:mip}) is based on $4$ sets of binary variables,
$1$ set of integer variables and $23$ types of constraints, $9$ of which use the Big-M
method. So many variables and constraints make implementation difficult, while the Big-M
method introduces a large penalty term that, if not chosen carefully, leads to serious
rounding errors and ill conditioning \cite[Section $5.4.2$]{griva2009}.

Second, there is no algorithm that is simultaneously scalable, distributed, and able to
solve the CFSTP in dynamic environments (Section \ref{sec:pstat}). To solve this issue, we
want to extend the CTS algorithm (Section \ref{sec:cts}), since it is anytime, has a
polynomial time complexity, and can be used in dynamic environments. Its only limitation
in relation to our objectives is to be centralised. In real-world domains such as disaster
response, this leads to problems such as single points of failure, unsustainable
computational loads, and poor performance in case of rapid changes of situation (Section
\ref{sec:pstat}).
To date, only \cite{ramchurn2010fms} have proposed a dynamic and distributed solution to
the CFSTP. They reduced it to a DynDCOP (Section \ref{sec:lit-dyndcop}) and solved it with
FMS, a variant of Max-Sum specialised for task allocation (Page \pageref{sec:lit-fms}).
However, unlike CTS, FMS is not guaranteed to converge, it is not anytime, and its
runtime is exponential in the number of agents. \cite{pujol2015} proposed BinaryMS
(Section \ref{sec:robotests}), another Max-Sum variant which, compared to FMS, lowers the
runtime to polynomial and achieves the same solution quality. Nonetheless, even BinaryMS
is not guaranteed to converge and not anytime. In addition, it requires a preprocessing
step with exponential runtime to transform the problem constraints into binary form,
which makes it not suitable for dynamic environments.

Lastly, no realistic CFSTP test framework has been proposed so far, although this
shortcoming was already identified in the original article \cite[Section
$8$]{ramchurn2010cfstp}. There are very few such frameworks even for the DynDCOP, to which
the CFSTP can be reduced, as mentioned above. Indeed, although the DCOP and DynDCOP models
can capture numerous real-world problems, researchers usually perform their empirical
evaluations on synthetic problems or classical combinatorial problems, such as graph
colouring and resource allocation \cite{fioretto2018survey}. To the best of our knowledge,
to date only the following (Dyn)DCOP works have conducted tests based on real-world data.
\cite{maheswaran2004b} considered resource-constrained multiple-event scheduling problems
occurring in office environments. \cite{junges2008} evaluated the performance of complete
DCOP algorithms in traffic light synchronisation problems. \cite{kim2011} developed
heuristics for applying Max-Sum to problems based on the real-time sensor system NetRad.
\cite{nelke2020,tkach2021} studied law enforcement problems inspired by police logs.
However, none of these test frameworks is large-scale\footnote{It is worth mentioning
\cite{leite2019a}, which is, to the best of our knowledge, the only study to date that
evaluates incomplete DCOP algorithms in large-scale problems, although not using
real-world data.}.
% shneider2020 also use real-world data, although their problem is a simpler MRTA

In the following sections, we address the above issues by proposing:
\begin{itemize}
    \item A novel mathematical programming formulation of the CFSTP, based only on binary
        variables and $5$ types of constraints, which do not use the Big-M method;
    \item A distributed version of the CTS algorithm that preserves its properties, namely
        being anytime, scalable and guaranteed to converge;
    \item The first large-scale, dynamic and distributed CFSTP test framework, based on
        real-world data published by the London Fire Brigade
        \cite{lfb-incident,lfb-mobilisation}.
\end{itemize}

\section{A Minimal Mathematical Program of the CFSTP}\label{sec:cfstp-bip}

We formulate the CFSTP as a \emph{Binary Integer Program} (BIP) \cite{wolsey2020}. Based
on the definitions of Sections \ref{sec:defs} $-$ \ref{sec:cv}, we detail our decision
variables, constraints and objective function.

\subsection{Decision variables}\label{sec:decvar}

Similar to Section \ref{sec:original-dv}, we use the following \emph{indicator} or binary
variables:
\begin{gather}
    \forall v \in V,\, \forall t \leq \gamma_v,\, \forall C \subseteq A,\;
    \mxc \in \left\{ 0, 1 \right\}\label{eq:bip0}\\
    \forall v \in V,\; y_v \in \left\{ 0, 1 \right\}\label{eq:bip0b}
\end{gather}
where: $\mxc = 1$ if coalition $C$ works on task $v$ at time $t$, and $0$ otherwise; $y_v
= 1$ if task $v$ is completed, and $0$ otherwise. Specifying indicator variables for
individual agents is not necessary, since they can be inferred from Equation
\ref{eq:bip0}.

\subsection{Constraints}\label{sec:bip-constraints}

There are $3$ types of constraints: structural, temporal and spatial.

\paragraph{Structural Constraints}

At each time, at most one coalition can work on each task:
\begin{equation}\label{eq:bip1}
    \forall v \in V,\, \forall t \leq
    \gamma_v,\; \sum_{C \subseteq A} \mxc \leq 1
\end{equation}
Equations \ref{eq:bip0} and \ref{eq:bip1} ensure that $\nexists \mxc : C \not\subseteq A$,
and that the maximum coalition size is $|A|$.

\paragraph{Temporal Constraints}

Tasks can be completed only by their deadlines, and no agent can work on a task after its
completion:
\begin{gather}
    \forall v \in V,\; y_v \leq 1\label{eq:bip2a}\\
    \forall v \in V,\;
    \left\lceil \sum_{t \leq \gamma_v}\, \sum_{C \subseteq A} u(C, v) \cdot \mxc
    \right\rceil = \left\lceil w_v \cdot y_v \right\rceil\label{eq:bip2b}
\end{gather}

\paragraph{Spatial Constraints}

An agent cannot work on a task before reaching its location. This identifies two cases:
when an agent reaches a task from its initial location, and when an agent moves from one
task location to another. The first case imposes that, for each task $v$, time $t \leq
\gamma_v$ and coalition $C$, the variable $\mxc$ can be positive only if all agents in $C$
can reach location $l_v$ at a time $t' < t$:
\begin{equation}\label{eq:bip3}
    \forall v \in V,\, \forall C \subseteq A,\;
    \text{if } \hat{\rho} = \max_{a \in C} \rho(a, l_a^0, l_v) \leq \gamma_v
    \text{ then }
    \sum_{t \leq \hat{\rho}}\, \mxc = 0
\end{equation}
where $\hat{\rho}$ is the maximum time at which an agent $a \in C$ reaches $l_v$, from its
initial location at time $t = 0$. Conditional constraints are usually formulated using
auxiliary variables or the Big-M method \cite{wolsey2020}.
However, such approaches further enlarge the mathematical program or can cause numerical
issues (Section \ref{sec:cfstp-gaps}). Consequently, in the preprocessing step necessary
to create our BIP, we can implement Equation \ref{eq:bip3} simply by excluding the
variables that must be equal to $0$. That is, if $\hat{\rho} \leq \gamma_v$, we only
declare the following variables: $\{ \mxc \}_{t \in [\hat{\rho} + 1, \gamma_v]}$.
The second case requires that if an agent cannot work on two tasks consecutively, then it
can work on at most one:
\begin{equation}\label{eq:bip4}
    \begin{gathered}
    \forall v_1, v_2 \in V,\,
    \forall C_1, C_2 \subseteq A \text{ such that } C_1 \cap C_2 \neq \emptyset,\\
    \forall t_1 \leq \gamma_{v_1},\, \forall t_2 \leq \gamma_{v_2} \text{ such that }
    t_1 + \max_{a \in C_1 \cap C_2} \rho(a, l_{v_1}, l_{v_2}) \geq t_2,\\
    x_{v_1,\, t_1, C_1} + x_{v_2,\, t_2, C_2} \leq 1
    \end{gathered}
\end{equation}
Hence, coalition $C_2$ can work on task $v_2$ only if all agents in $C_1 \cap C_2$ can
reach location $l_{v_2}$ by deadline $\gamma_{v_2}$. Equation \ref{eq:bip4} also implies
that an agent cannot work on multiple tasks at the same time.

There are no synchronisation constraints \cite{nunes2017taxonomy}. Thus, when a task $v$
is allocated to a coalition $C$, each agent $a \in C$ starts working on $v$ as soon as it
reaches its location, without waiting for the remaining agents. This means that $v$ is
completed by a temporal sequence of subcoalitions of $C$: $\exists S \subseteq P(C)$ such
that $\forall C' \in S$, $\exists t \leq \gamma_v$, $x_{v,\, t,\, C'} = 1$, where $P(C)$
is the power set of $C$.

\subsection{Objective Function}\label{sec:bip-of}

Let $\bm{x}$ be a \emph{solution}, that is, a value assignment to all variables, which
defines the route and schedule of each agent. The objective is to find a solution that
maximises the number of completed tasks:
\begin{equation}\label{eq:bip}
    \arg \max_{\bm{x}} \sum_{v \in V}\, y_v
    \text{ subject to Equations \ref{eq:bip0} $-$ \ref{eq:bip4}}
\end{equation}

Both creating all decision variables (Section \ref{sec:decvar}) and finding an optimal
solution exhaustively (Equation \ref{eq:bip}) may require to list all possible
coalition allocations for each possible permutation of $V$, with a worst-case time
complexity of:
\begin{equation}\label{eq:comp}
    O \left( |V|! \cdot 2^{|A|} \cdot \tmax \right)
\end{equation}
which is the same as the CFSTP formulation of Section \ref{sec:of}.

\begin{theorem}\label{teo:eq}
    Equation \ref{eq:bip} is equivalent to the original MIP of the CFSTP (Section
    \ref{sec:mip}).
\end{theorem}
\begin{proof}
    Since we use the original objective function (Equation \ref{eq:cfstpof}), below we
    show how our constraints imply the original ones (Equations \ref{eq:mip0} $-$
    \ref{eq:mip22}).

    Completing tasks by their deadlines and allowing agents to work only on uncompleted
    tasks implies that the total work done for each task $v$ is equal to $w_v$ if $v$ is
    completed, and $0$ otherwise. Hence, Equations \ref{eq:bip2a}, \ref{eq:bip2b}
    $\Rightarrow$ Equations \ref{eq:mip0}, \ref{eq:mip1}.

    Equation \ref{eq:bip1} is equivalent to Equation \ref{eq:mip2}.

    If at most one coalition can work on each task at each time, and an agent cannot
    work on a task before reaching its location, then each agent can work on at most one
    task at each time. Consequently, Equations \ref{eq:bip1}, \ref{eq:bip3} $\Rightarrow$
    Equation \ref{eq:mip6}.

    Equation \ref{eq:mip3} is not necessary because $t \leq \gamma_v$ for each $\mxc$
    (Equation \ref{eq:bip0}).

    If agent $a$ cannot work on task $v$ before reaching its location $l_v$, then it can
    do so after finishing work on a previous task, or after reaching $l_v$ from its
    initial location $l_0^a$. Furthermore, since the objective is to maximise the number
    of completed tasks, if $v$ is allocated to $a$, then $a$ changes the status of its
    service (from \emph{free} to \emph{working} and vice versa) exactly twice, and the
    decision variables related to $a$ and $v$ are equal to $1$ only when $a$ works on $v$,
    and $0$ otherwise. Thus, Equations \ref{eq:bip3} $-$ \ref{eq:bip} $\Rightarrow$
    Equations \ref{eq:mip4}, \ref{eq:mip5}, \ref{eq:mip7} $-$ \ref{eq:mip9}.

    If agent $a$ cannot work on two tasks consecutively, then it can work on at most one.
    Therefore, $a$ cannot leave and reach the same task location, it can only reach a new
    task location from exactly one location, and it can only leave a location to reach
    exactly one task location. That is, Equation \ref{eq:bip4} $\Rightarrow$ Equations
    \ref{eq:mip10} $-$ \ref{eq:mip12}.

    Since task $v$ can only be completed by deadline $\gamma_v$, agent $a$ can work on $v$
    only if it can reach its location $l_v$ by $\gamma_v$, and the objective is to
    maximise the number of completed tasks, then: the coalition $C$ of which $a$ is a
    member reaches $v$ only to complete it; $C$ works for at least $1$ unit of time on $v$
    (assuming that $w_v \geq 1$, $\forall v \in V$), and each $a \in C$ reaches $l_v$ from
    another task location or from its initial location $l_0^a$. Thus, Equations
    \ref{eq:bip2b} $-$ \ref{eq:bip4}, \ref{eq:bip} $\Rightarrow$ Equations \ref{eq:mip15} $-$
    \ref{eq:mip20}.

    Equation \ref{eq:bip1} is equivalent to Equation \ref{eq:mip21}.

    Since the original MIP assumes that the allocation process starts at $t = 1$ (Section
    \ref{sec:original-dv}), Equation \ref{eq:bip3} $\Rightarrow$ Equation \ref{eq:mip22}. If
    we remove this assumption, then Equation \ref{eq:mip22} is not necessary.
\end{proof}
Having significantly fewer constraints than the original MIP, our BIP can be used more
effectively by exact algorithms based on branch-and-cut or branch-and-price \cite[Section
$3.1.1$]{top2019}. A trivial way to solve the CFSTP would be to implement Equation
\ref{eq:bip} with solvers such as CPLEX or GLPK. Although this would guarantee anytime and
optimal solutions, it would also take exponential time to both create and solve our BIP
(Equation \ref{eq:comp}). This limits this practice to offline contexts and very small
problems. For example, using CPLEX $20.1$ with commodity hardware, and the test setup of
\cite{ramchurn2010cfstp}, we can solve problems where $|A| \cdot |V| \leq 20$ in hours.
With bigger problems, CPLEX depletes all memory ($8$ GB) and crashes.

Another major issue with centralised generation of optimal solutions is that, in real-time
domains such as disaster response, it can be computationally not feasible or economically
undesirable, especially when the problem changes frequently (Section \ref{sec:pstat}). For
these reasons, the next section presents a scalable, dynamic and distributed algorithm to
solve the CFSTP.

\section{A Scalable, Dynamic and Distributed CFSTP Algorithm}\label{sec:dcts}

We propose a novel reduction of the CFSTP to a DynDCOP, then we show how CTS can solve the
problem in this formulation. We use the DynDCOP formalism because it has proven largely
capable of modelling disaster response problems (Section \ref{chap2:choice}).

\subsection{Reduction of the CFSTP to a DynDCOP}\label{sec:reduction}

Using Definition \ref{def:dyndcop}, we formalise a DynDCOP as a sequence $\mathcal{D} =
\left\{ \mathcal{D}_t \right\}_{t \leq \tmax}$, where each $\mathcal{D}_t = (A^t, X^t,
D^t, F^t)$ is a DCOP such that $A^t \subseteq A$, and:
\begin{itemize}
    \item $X^t = \{\chi_1^t, \dots, \chi_k^t\}$ is a set of $k = |A^t| \leq n$ variables,
        where $\chi_i^t$ indicates the task performed by agent $a_i^t \in A^t$;
    \item $D^t = \{D_1^t, \dots, D_k^t\}$ is a set of $k$ variable domains, such that
        $\chi_i^t \in D_i^t$. A set $d = \{d_1, \dots, d_k\}$, where $d_i \in D_i^t$, is
        called an \emph{assignment}. Each $d_i \in d$ is called the $i$-th \emph{variable
        assignment} and is the value assigned to variable $\chi_i^t$;
    \item $F^t = \{f_1^t, \dots, f_h^t\}$ is a set of $h \leq m$ functions, where $f_i^t$
        represents the constraints on task $v_i^t$. In particular, each $f_i^t : D_{i_1}^t
        \times \cdots \times D_{i_{h_i}}^t \to \mathbb{R}_{\geq 0}$ assigns a non-negative
        real cost to each possible assignment to the variables $X_{h_i}^t \subseteq X^t$,
        where $h_i \leq h$ is the arity of $f_i^t$.
\end{itemize}
The objective of $\mathcal{D}$ is to find an assignment that minimises all costs:
\begin{equation}\label{eq:dcop}
    \forall t \leq \tmax,\, \arg \min_{d \in D^t}
    \sum_{f_i^t \in F^t} f_i^t(d_{i_1}, \dots, d_{i_{h_i}})
\end{equation}
It is typically assumed that if $\chi_i^t$ is in the scope of $f_j^t$, then agent $a_i^t$
knows $f_j^t$ \cite[Section $4.2$]{fioretto2018survey}. To reduce the CFSTP to a DynDCOP,
we specify $A^t$, $D^t$ and $F^t$ as follows. At time $t$, let $A^t$ be the set of free
agents (Section \ref{sec:cfla1}), and let $V^t_{allocable}$ be the set of tasks that have
not yet been completed. The domain of each variable $\chi_i^t$ is:
\begin{equation}\label{eq:dyndcop-d}
    D_i^t = \left\{ v \in V^t_{allocable} \text{ such that } t + \rho(a_i^t,
    l_{a_i^t}, l_v) \leq \gamma_v \right\} \cup \left\{ \varnothing \right\}
\end{equation}
where $\varnothing$ means that no task is allocated to agent $a_i^t$. Hence, $A^t$
satisfies the structural constraints, while $D_i^t$ contains all tasks that at time $t$
can be allocated to $a_i^t$ satisfying the spatial constraints (Section
\ref{sec:bip-constraints}). Let $\bm{x}_i \subseteq \bm{x}$ be a \emph{singleton
solution}, that is, a solution to task $v_i$ (Section \ref{sec:bip-of}). At time $t$, let
$\bm{x}_i^t \subseteq \bm{x}_i$ be a singleton solution corresponding to
$f_i^t(d_{i_1}, \dots, d_{i_{h_i}})$, defined as follows. Each $x_{v_i, \, t,\, C} \in
\bm{x}_i^t$ is such that $C$ is a subset of the agents that control the variables in
the scope of $f_i^t$, while $x_{v_i,\, t,\, C} = 1$ if $d_{i_{h_i}} = v_i$, for each
$h_i$-th agent in $C$, and $0$ otherwise. To satisfy the temporal constraints (Section
\ref{sec:bip-constraints}), each $i$-th function is defined as follows:
\begin{equation}\label{eq:costs}
    f_i^t(d_{i_1}, \dots, d_{i_{h_i}}) =
    \min_{\bm{x}_i^t,\, t' \leq \gamma_{v_i}}\, \left\lceil \sum_{s \leq t',\,
    x_{v_i,\, s,\, C} \in \bm{x}_i^t} u(C, v) \right\rceil = \left\lceil w_v \right\rceil
\end{equation}
with the convention that $f_i^t(d_{i_1}, \dots, d_{i_{h_i}}) = +\infty$ if $v_i$ cannot be
completed by deadline $\gamma_{v_i}$. Hence, the solution space of $\mathcal{D}$ satisfies
all CFSTP constraints, while minimising all costs implies minimising the time required to
complete each task (Equations \ref{eq:dcop} and \ref{eq:costs}), which implies maximising
the number of completed tasks, as required by the objective function of the CFSTP
(Equation \ref{eq:bip}).

\subsection{Distributed CTS}

Before introducing our distributed version of CTS (Section \ref{sec:cts}), we present the
data structure and the communication protocol on which it is based.

\begin{figure}[t]
    \centering
    \begin{adjustbox}{width=.45\textwidth}
        \begin{tikzpicture}
    \begin{scope}[square/.style={regular polygon,regular polygon sides=4}]
        \node (v1) at (-3, 0) [square,draw] {$f_1$};
        \node (v2) at (-1.5 , -1.5) [square,draw] {$f_2$};
        \node (v3) at (0 , 0) [square,draw] {$f_3$};
        \node (v4) at (1.5 , -1.5) [square,draw] {$f_4$};
        \node (a1) at (-1.5, 0) [circle,draw] {$\chi_1$};
        \node (a2) at (0 , -1.5) [circle,draw] {$\chi_2$};
    \end{scope}
    \begin{scope}
        \draw (a1) -- (v1);
        \draw (a1) -- (v2);
        \draw (a1) -- (v3);
        \draw (a2) -- (v2);
        \draw (a2) -- (v3);
        \draw (a2) -- (v4);
    \end{scope}
\end{tikzpicture}

    \end{adjustbox}
    \caption[An example factor graph]{The factor graph of a DCOP with $2$ agents and $4$
        tasks. In our formulation, a DCOP represents the state of a CFSTP at a certain
        time, in which circles are variables of free agents, squares are cost functions of
        uncompleted tasks, and each edge connects an agent to a task it can reach by its
        deadline.}
    \label{fig:fgex}
\end{figure}

To represent DCOPs, we use factor graphs (Definition \ref{def:fg}). As an example, Figure
\ref{fig:fgex} shows the factor graph of the function $F(X) = f_1(\chi_1) + f_2(\chi_1,
\chi_2) + f_3(\chi_1, \chi_2) + f_4(\chi_2)$. In a factor graph $G$, a solution is found
by allowing nodes to exchange messages. Hence, to execute CTS on $G$, we have to define
how nodes communicate and operate. Below, we present a communication protocol and
algorithms for both variable and factor nodes. Based on the well-established formalism of
\cite{yokoo1992}, the nodes communicate in the following way:
\begin{itemize}
    \item Node $i$ can message node $j$ only if $i$ knows the address of $j$\footnote{For
        instance, the IP address of $j$, if the nodes are connected to the same network.}.
        In our context, if $\chi_i^t$ is in the scope of $f_j^t$, then $\chi_i^t$ knows
        the address of $f_j^t$, and vice versa;
    \item Each node $i$ has a message queue $Q_i$, to which messages are delivered with a
        finite delay;
    \item Node $i$ can use the function \textsc{receive}() to dequeue a message from
        $Q_i$, and the function \textsc{send}($j$, \texttt{illoc\_force}, \texttt{[args]})
        to send a message to $j$. Node $j$ will receive a message in the format
        \texttt{(sender, illoc\_force, [args])}, where \texttt{sender} is the identifier
        of node $i$, \texttt{illoc\_force} is its illocutionary force, and \texttt{[args]}
        is an optional list of arguments. An \emph{illocutionary force} is either an
        information or a command \cite{vieira2007}.
\end{itemize}
We assume that the node of each function is controlled by an agent in its
scope\label{fgas}. This agent can be chosen randomly or according to some criterion.
Moreover, we assume that each agent can independently retrieve the system time (i.e., the
current value of $t$) in constant time.

Algorithm \ref{algo:d1} presents the operation of variable node $\chi_i^t$. If there is
an uncompleted task $v_j^t$ that can be allocated to free agent $a_i^t$ (lines $1 - 3$),
then variable node $\chi_i^t$ communicates to factor node $f_j^t$ the ability of $a_i^t$
to work on $v_j^t$, also specifying the time at which it can reach and start working on it
(lines $4 - 6$). After that, it waits until it gets a reply from $f_j^t$ or a
predetermined time interval expires (lines $7 - 9$). If it receives the approval of
$f_j^t$, then $v_j^t$ is allocated to $a_i^t$ (lines $10 - 11$). At line $2$, $v_j^t$ is
chosen such that it is the closest to $a_i^t$ and $\gamma_{v_j^t}$ is the earliest
deadline. Phase $1$ is completed after that each $\chi_i^t$ executes line $6$.

\begin{algorithm}[t]
    \DontPrintSemicolon
    $\chi_i^t \gets \varnothing$ \Comment*[h]{\small the agent is initially \emph{free}}\;
    $d_j \gets$ get task allocable to agent $a_i^t$ at time $t$ \Comment*[h]{\small
    Algorithm \ref{algo:getTask}}\;
    \If{$d_j \neq \varnothing$}{
        $s_i \gets$ time at which agent $a_i^t$ can start working on task $d_j$\;
        $f_j^t \gets$ factor node of $d_j$\;
        %create edge to $f_j^t$\;
        \textsc{send}($f_j^t$, \textsf{assignable}, $s_i$)\;
        \textsf{msg} $\gets$ \textsc{nil}\;
        \While{\normalfont\textsf{msg} not received from $f_j^t$ or not time out}{
            \textsf{msg} $\gets$ \textsc{receive}()\;
        }
        \If{\normalfont\textsf{msg} $= (f_j^t, \text{\textsf{allocate}})$}{
            $\chi_i^t \gets d_j$\;
        }
    }
    \caption{CTS node of variable $\chi_i^t$\label{algo:d1}}
\end{algorithm}

Algorithm \ref{algo:d2} presents the operation of factor node $f_j^t$. The loop at lines
$1 - 2$ is a synchronisation step that allows $f_j^t$ to know which agents in its
neighbourhood can work on $v_j^t$. Lines $3 - 6$ enacts Phase $2$, while lines $7 - 9$
update workload $w_{v_j}$.
Hence, our factor graphs are synchronous networks, in which the factor nodes are the
synchronisers \cite{lynch1996}.

\begin{algorithm}[t]
    \DontPrintSemicolon
    \While{\normalfont not all neighbours sent an \textsf{assignable} message or not time out}{
        \textsf{msg} $\gets$ \textsc{receive}()\;
    }
    $\Pi^t_{v_j} \gets$ list of all assignable agents sorted by arrival time to $v_j$\;
    %\Comment*[h]{\small same as line $19$ of Algorithm \ref{algo:cts}}\;
    $C^* \gets$ minimum coalition in $\Pi^t_{v_j}$ that can complete $v_j$ by $\gamma_{v_j}$
    \Comment*[h]{\small Equation \ref{eq:costs}}\;
    \For{$a_i^t \in C^*$}{
        \textsc{send}($\chi_i^t$, \textsf{allocate})\;
    }
    $C^t_{v_j} \gets$ all agents working on $v_j$ at time $t$\;
    \If{$C^t_{v_j} \neq \emptyset$}{
        $w_{v_j} \gets w_{v_j} - u(C^t_{v_j}, v_j)$\;
    }
    \caption{CTS node of factor $f_j^t$\label{algo:d2}}
\end{algorithm}

We call \emph{Distributed CTS} (D-CTS) the union of Algorithms \ref{algo:d1} and
\ref{algo:d2}. Each message has size $O(1)$, since it always contains a node
address, a message flag and an integer. At time $t$, each variable node $\chi_i^t$ sends
at most $1$ message (line $6$ in Algorithm \ref{algo:d1}), while each factor node $f_j^t$
sends $O(|A|)$ messages (lines $5 - 6$ in Algorithm \ref{algo:d2}). Assuming that all
tasks can be completed, the total number of messages sent is:
\begin{equation}\label{eq:dcts-co}
    O(|A| + |V| \cdot |A|) = O(|V| \cdot |A|)
\end{equation}
The runtime of Algorithm \ref{algo:d1} is $O(|V|)$, because line $2$ selects a task in the
neighbourhood of an agent. The runtime of Algorithm \ref{algo:d2} is $O(|A| \log |A|)$,
due to the sorting at line $3$ \cite{cormen2009}.
Since both algorithms are executed up to $\tmax$ times, the overall time complexity of
D-CTS is the same as CTS (Equations \ref{eq:comp1} and \ref{eq:comp2}).
The advantages of D-CTS are:
\begin{enumerate}
    \item
        It is anytime, since it decomposes a CFSTP into a set of subproblems (Section
        \ref{sec:cts}). This property is not trivial to guarantee in distributed systems
        \cite{zivan2014}, and is missing in main DCOP algorithms, such as ADOPT, DPOP,
        OptAPO and Max-Sum (Table \ref{t:dcop_chars});
    \item
        It is self-stabilising (Definition \ref{def:self-stabilising}), being guaranteed
        to converge (Theorem \ref{teo:convergence}), and given that each agent can only
        work on a new task after completing the one to which it is currently assigned
        (Algorithm \ref{algo:d1});
    \item
        The phase-based design has $2$ performance benefits. First, the algorithm is not
        affected by the structure of factor graphs. For instance, in a cyclic graph like
        the one in Figure \ref{fig:fgex}, where the same $a > 1$ tasks can be allocated to
        the same $b > 1$ agents, inference-based DCOP algorithms (e.g., Max-Sum variants)
        in general are not guaranteed to converge, unless they are augmented with specific
        techniques (e.g., damping or ADVP). Second, the algorithm is robust to
        \emph{disruptions}, that is, the addition or removal of factor graph nodes
        \cite[Section $6.2$]{ramchurn2010fms}. Disruptions are typical of dynamic
        environments (Section \ref{sec:pstat}). For instance, in disaster response, tasks
        are removed if some victims have perished, and are added if new fires are
        discovered. Likewise, new agents can be added to reflect the availability of
        additional workforce, while existing ones are removed when they deplete their
        resources, or are unable to continue due to sustained damages. Unlike D-CTS, the
        majority of DCOP algorithms (e.g., Max-Sum and DPOP) cannot handle disruptions,
        unless they are properly augmented (e.g., FMS and S-DPOP);
    \item
        Unlike most DCOP algorithms (e.g., ADOPT and DPOP), the communication overhead
        (i.e., the number of messages exchanged) is at most linear (Equation
        \ref{eq:dcts-co}), and each agent does not need to maintain an information graph
        of all other agents;
    \item
        Finally, performance does not depend on any tuning parameters, as is the case with
        other DCOP algorithms (e.g., DSA, AED and DPSA).
\end{enumerate}
Since each factor node is controlled by an agent (Page \pageref{fgas}), D-CTS is not fully
distributed, but \emph{partially centralised}. However, this is a typical assumption in
DCOP algorithms that use factor graphs or pseudo-trees, as it allows to explicitly handle
$k$-ary constraints \cite[Section $4.2$]{fioretto2018survey}. This is also the case for
main algorithms such as AFB, ADOPT, DPOP and Max-Sum (Section \ref{sec:dcop0}).
Partial centralisation improves local coordination and thus overall performance, but
reduces privacy \cite[Section $4.3.1$]{fioretto2018survey}.
In cooperative coordination, and particularly in disaster response, this is generally not
an issue.

Algorithms \ref{algo:d1} and \ref{algo:d2} resemble a \emph{single-item auction}
\cite{dias2006}, where, for each time $t$ and task $v_j$: the \emph{bidders} are the
agents in $A^t$ that can reach $v_j$ by $\gamma_{v_j}$; the \emph{bid} of an agent is the
time at which it can start working on $v_j$; the \emph{auctioneer} is the agent
controlling factor node $f^t_j$, which closes the auction by sending \textsf{allocate}
messages to selected agents.
However, there are two differences from a classic auction. First, $v_j$ can be
allocated to more than one agent at once.
Second, agents cannot be overburdened with evaluation problems, since each bidder is
interested in at most one task, and therefore each auctioneer receives as few bids as
possible. In other words, the communication overhead is minimised.
This advantage is due to our reduction of the CFSTP to a DynDCOP (Section
\ref{sec:reduction}), in which the search space contains only feasible solutions.

\section{Empirical Evaluation in Dynamic Environments}\label{sec:dcts-tests}

We created a dataset\footnote{\url{https://doi.org/10.5281/zenodo.4728012}} with $347588$
tasks using open records published by the \emph{London Fire Brigade} (LFB) over a period
of $11$ years. Then, we wrote a test framework in
Java\footnote{\url{https://doi.org/10.5281/zenodo.4764646}} and compared D-CTS with
DSA-SDP, a state-of-the-art incomplete, synchronous and search-based DCOP algorithm
(Page \pageref{sec:dsa}).

We adapted DSA-SDP to solve our DynDCOP formulation (Section \ref{sec:reduction}), which
finds a CFSTP solution by solving multiple DCOPs. Hence, being a DCOP algorithm, its
performance is not penalised in our test framework. We chose it as our baseline because,
similarly to D-CTS, it has a polynomial coordination overhead and is scalable (Table
\ref{t:dcop_chars}). We kept the parameters of \cite{zivan2014} and ran
$|V^t_{allocable}|$ iterations at each time $t$, since we found that, in our test
framework, running more iterations can only marginally improve the solution quality, while
requiring a significant increase in communication overhead and time complexity. Below, we
detail our setup and discuss the results.

\subsection{Setup}\label{sec:setup}

Let $\mathcal{N}$ and $\mathcal{U}$ denote the normal and uniform distribution,
respectively. A test configuration consists of the following parameters:
\begin{itemize}
    \item Since there are currently $150$ identical London fire engines in operation, $|A|
        = 150$ for each problem. All agents have the same speed;
        % It is reasonable to assume a fixed speed, because if an algorithm is better than
        % others with homogeneous agent speeds, then it is better with heterogeneous agent
        % speeds as well.
    \item $|V| = |A| \cdot k$, where $k \in \{ 1, \dots, 20 \}$. Thus, problems have up to
        $3000$ tasks;
    \item The demand of each task $v$ is defined by a record dated between $1$ January
        $2009$ and $31$ December $2020$. More precisely, $\gamma_v$ is the attendance time
        (in seconds) of the firefighters, and since the median attendance time in the
        whole dataset is about $5$ minutes, we set $w_v \sim \mathcal{U}(10, 300)$ to
        simulate wide-ranging workloads;
    \item For each task-to-agent ratio $|V|/|A|$, the nodes of a problem are chosen in
        chronological order. That is, the first problem always starts with record $1$, and
        if a problem stops at record $q$, then the following one will use records $q + 1$
        to $q + 1 + |V|$;
    \item The locations are latitude-longitude points, and the travel time $\rho(a, l_1,
        l_2)$ is given by the great-circle distance in kilometers between locations $l_1$
        and $l_2$, divided by the (fixed) speed of agent $a$;
    \item In addition to task locations, $L$ contains the locations of the $103$ currently
        active London fire stations. In each problem, each agent starts at a fire station
        defined by the record of a task.
\end{itemize}
Regardless of its individual features, each agent may perform differently in different
coalitions, due to the interaction with other agents. To generate coalition values, we
start by taking from \cite[Section $4$]{rahwan2012} the following well-known CSG
distributions:
\begin{enumerate}
    \item \emph{Normally Distributed Coalition Structures} (NDCS): $u(C, v) \sim
        \mathcal{N}(|C|, \sqrt[4]{|C|})$;
    \item \emph{Agent-based}: each agent $a$ has a value $p_a \sim \mathcal{U}(0,
        10)$ representing its individual performance and a value $p_a^C \sim
        \mathcal{U}(0, 2 \cdot p_a)$ representing its performance in coalition
        $C$. The value of a coalition is the sum of the values of its members:
        $u(C, v) = \sum_{a \in C} p_a^C$.
\end{enumerate}
Then, we decrease each $\mu_v = u(C, v)$ by values $r$ and $q$, both sampled from
$\mathcal{U}(\mu_v / 10, \mu_v / 4)$ with probability $\gamma_v/(\tmax + 1)$ and $|C|/(|A|
+ 1)$, respectively. The perturbation $r$ simulates real-time domains where the later
$\gamma_v$ is, the lower the benefit of performing $v$ \cite{stankovic2013edf}. The
perturbation $q$ simulates situations where the more agents there are, the greater the
likelihood of congestion and thus of reduced performance, as it can happen in large-scale
robot swarms \cite{guerrero2017}. We call the resulting distributions UC\_NDCS and
UC\_Agent-based, where UC means \emph{Urgent and Congested}. NDCS assigns lower values to
solutions containing fewer coalitions \cite{rahwan2009}, while Agent-based does the
opposite by definition. Both distributions are neither superadditive nor subadditive
\cite{rahwan2015survey}. Hence, it is not possible to define a priori an optimal coalition
for each task.
We ensured consistency between the results of the algorithms by computing and storing
coalition values in hash maps. More precisely, the maps were lazy-initialised and shared
among all problems.

During the solution of each problem, we gradually removed agents to simulate
\emph{degradation} scenarios. The removal rate was calculated with a Poisson cumulative
distribution function $Pois_{CDF}(\bm{a}, \lambda)$, where $\bm{a}$ contains all
firefighter arrival times in the dataset, and the rate $\lambda$ is the average number of
incidents per hour and per day. For each test configuration and algorithm, we solved
$100$ problems and measured the median and $95\%$ confidence interval of: number of
messages sent; \emph{network load}, or the total size of messages sent; number of
\emph{Non-Concurrent Constraint Checks} (NCCCs) \cite{meisels2007}; percentage of tasks
completed, and CPU time\footnote{Based on an Intel Xeon E5-2670 processor ($2.6$ GHz, $8$
threads).}.
% We do not measure problem completion times, because that makes sense only in static
% environments.

\subsection{Results}\label{sec:rres}

\begin{figure}[t]
    \centering
    \begin{adjustbox}{width=\textwidth}
        \begin{tikzpicture}
    \begin{groupplot}[%
        group style={group size=2 by 1, horizontal sep=2cm},
        grid=both,
        grid style={line width=.1pt, draw=gray!10},
        major grid style={line width=.2pt,draw=gray!50},
        xtick={1,4,8,12,16,20},
        every axis plot/.append style={thick},
        legend pos=north east]

        \large

        % Completed tasks

        \nextgroupplot[xlabel=$|V| / |A|$,ylabel=Completed tasks ($\%$),
        title=(a) UC\_NDCS, ymin=0, ymax=100, legend entries={\textsc{d-cts},
        \textsc{dsa-sdp}}]

        \addlegendimage{blue, line width=1.1}
        \addlegendimage{red,dashed, line width=1.1}

        %% DSA-SDP
        \addplot [dashed, red, line width=1.1] coordinates {%
            (1, 50)
            (2, 30)
            (3, 22)
            (4, 18)
            (5, 14)
            (6, 11)
            (7, 10)
            (8, 9)
            (9, 8)
            (10, 7)
            (11, 7)
            (12, 6)
            (13, 6)
            (14, 6)
            (15, 5)
            (16, 5)
            (17, 4)
            (18, 4)
            (19, 4)
            (20, 4)
        };

        \addplot [name path=u1, draw=none] plot coordinates {%
            (1, 54)
            (2, 32)
            (3, 24)
            (4, 19)
            (5, 15)
            (6, 12)
            (7, 11)
            (8, 10)
            (9, 9)
            (10, 8)
            (11, 7)
            (12, 7)
            (13, 6)
            (14, 6)
            (15, 5)
            (16, 5)
            (17, 5)
            (18, 4)
            (19, 4)
            (20, 4)
        };

        \addplot [name path=l1, draw=none] plot coordinates {%
            (1, 46)
            (2, 29)
            (3, 21)
            (4, 16)
            (5, 13)
            (6, 11)
            (7, 10)
            (8, 9)
            (9, 8)
            (10, 7)
            (11, 6)
            (12, 6)
            (13, 6)
            (14, 5)
            (15, 5)
            (16, 4)
            (17, 4)
            (18, 4)
            (19, 3)
            (20, 3)
        };

        \addplot+ [fill=red!25,postaction={pattern color = red!40, pattern=north
        west lines}] fill between[of=u1 and l1];

        %% D-CTS
        \addplot [blue, line width=1.1] coordinates {%
            (1, 99)
            (2, 51)
            (3, 34)
            (4, 27)
            (5, 23)
            (6, 19)
            (7, 16)
            (8, 15)
            (9, 13)
            (10, 12)
            (11, 11)
            (12, 11)
            (13, 10)
            (14, 9)
            (15, 8)
            (16, 8)
            (17, 7)
            (18, 7)
            (19, 6)
            (20, 6)
        };

        \addplot [name path=u2, draw=none] plot coordinates {%
            (1, 100)
            (2, 52)
            (3, 36)
            (4, 28)
            (5, 24)
            (6, 20)
            (7, 17)
            (8, 16)
            (9, 14)
            (10, 13)
            (11, 12)
            (12, 11)
            (13, 10)
            (14, 10)
            (15, 9)
            (16, 8)
            (17, 8)
            (18, 7)
            (19, 7)
            (20, 6)
        };

        \addplot [name path=l2, draw=none] plot coordinates {%
            (1, 98)
            (2, 50)
            (3, 34)
            (4, 26)
            (5, 21)
            (6, 18)
            (7, 16)
            (8, 15)
            (9, 13)
            (10, 12)
            (11, 11)
            (12, 10)
            (13, 9)
            (14, 9)
            (15, 8)
            (16, 7)
            (17, 7)
            (18, 6)
            (19, 6)
            (20, 5)
        };

        \addplot+ [fill=blue!25,opacity=0.6] fill between[of=u2 and l2];

        \nextgroupplot[xlabel=$|V| / |A|$,ylabel=Completed tasks ($\%$),
        title=(b) UC\_Agent-based, ymin=0, ymax=100,legend
        entries={\textsc{d-cts}, \textsc{dsa-sdp}}]

        \addlegendimage{blue,line width=1.1}
        \addlegendimage{red,dashed, line width=1.1}

        %% DSA-SDP
        \addplot [dashed, red, line width=1.1] coordinates {%
            (1, 58)
            (2, 35)
            (3, 26)
            (4, 21)
            (5, 17)
            (6, 15)
            (7, 13)
            (8, 11)
            (9, 10)
            (10, 9)
            (11, 9)
            (12, 8)
            (13, 7)
            (14, 7)
            (15, 7)
            (16, 6)
            (17, 6)
            (18, 5)
            (19, 5)
            (20, 4)
        };

        \addplot [name path=u1, draw=none] plot coordinates {%
            (1, 60)
            (2, 38)
            (3, 27)
            (4, 22)
            (5, 18)
            (6, 16)
            (7, 14)
            (8, 13)
            (9, 11)
            (10, 10)
            (11, 9)
            (12, 9)
            (13, 8)
            (14, 7)
            (15, 7)
            (16, 7)
            (17, 6)
            (18, 6)
            (19, 5)
            (20, 5)
        };

        \addplot [name path=l1, draw=none] plot coordinates {%
            (1, 57)
            (2, 34)
            (3, 25)
            (4, 20)
            (5, 16)
            (6, 14)
            (7, 12)
            (8, 11)
            (9, 10)
            (10, 9)
            (11, 8)
            (12, 7)
            (13, 7)
            (14, 6)
            (15, 6)
            (16, 6)
            (17, 5)
            (18, 4)
            (19, 4)
            (20, 4)
        };

        \addplot+ [fill=red!25,postaction={pattern color = red!40, pattern=north
        west lines}] fill between[of=u1 and l1];

        %% D-CTS
        \addplot [blue,line width=1.1] coordinates {%
            (1, 98)
            (2, 49)
            (3, 36)
            (4, 28)
            (5, 23)
            (6, 20)
            (7, 17)
            (8, 16)
            (9, 14)
            (10, 13)
            (11, 11)
            (12, 11)
            (13, 10)
            (14, 9)
            (15, 9)
            (16, 8)
            (17, 8)
            (18, 7)
            (19, 7)
            (20, 6)
        };

        \addplot [name path=u2, draw=none] plot coordinates {%
            (1, 98)
            (2, 52)
            (3, 37)
            (4, 29)
            (5, 25)
            (6, 21)
            (7, 18)
            (8, 16)
            (9, 14)
            (10, 13)
            (11, 12)
            (12, 11)
            (13, 10)
            (14, 9)
            (15, 9)
            (16, 8)
            (17, 8)
            (18, 7)
            (19, 7)
            (20, 7)
        };

        \addplot [name path=l2, draw=none] plot coordinates {%
            (1, 96)
            (2, 49)
            (3, 35)
            (4, 27)
            (5, 22)
            (6, 19)
            (7, 16)
            (8, 15)
            (9, 14)
            (10, 12)
            (11, 11)
            (12, 10)
            (13, 9)
            (14, 9)
            (15, 8)
            (16, 8)
            (17, 7)
            (18, 7)
            (19, 6)
            (20, 6)
        };

        \addplot+ [fill=blue!25, opacity=0.6] fill between[of=u2 and l2];
    \end{groupplot}
\end{tikzpicture}

    \end{adjustbox}
    \caption[Comparison between D-CTS and DSA-SDP]{Comparison between D-CTS and DSA-SDP in
        our test framework. Each subfigure denotes a coalition value distribution, while
        each point is the median and $95\%$ confidence interval over $100$ problems of the
        percentage of tasks completed. The X-axis is the task-to-agent ratio.}
    \label{fig:dcts-t}
\end{figure}

% UC_AGENT_BASED 1.35 \pm [0.29, 0.03] -> 3.03% \pm [34.98, 1.2]
% UC_NDCS 1.6 \pm [0.26, 0.06] -> 4.76% \pm [41.24, 2.76]
% Overall 1.52 \pm [0.33, 0.21] -> 3.79% \pm [42.22, 1.96]
Figure \ref{fig:dcts-t} and \ref{fig:dcts-t2} show our results. D-CTS completes $3.79\%
\pm [42.22\%, 1.96\%]$ more tasks than DSA-SDP (Figure \ref{fig:dcts-t}). For both
algorithms, the performance drops rapidly as the task-to-agent ratio increases. This is
due to the Urgent component in the coalition value distributions: the higher the ratio,
the higher the median task completion time.
Conversely, the Congested component can reduce the percentage of tasks completed more in
problems with smaller task-to-agent ratios, where agents can form larger coalitions and
thus increase the likelihood of congestion.

The network load of DSA-SDP is $0.59 \pm [0.41, 0.02]$ times that of D-CTS (Figure
\ref{fig:dcts-t2}b). This is because a DSA-SDP message contains only a task address,
while a D-CTS message also contains a binary flag and an integer (Section
\ref{sec:dcts}). In Java, an address requires $8$ bytes, a flag requires $1$ byte, and
an integer requires $1 - 4$ bytes. Hence, while a DSA-SDP message always requires $8$
bytes, a D-CTS message requires $10 - 13$ bytes. This is line with the results obtained.
However, the situation would be reversed if we performed $1000$ DSA-SDP iterations as
suggested in \cite{zhang2005}, since $median(\{ |V^t_{allocable}| \}_{t \leq \tmax}) \ll
1000$ in our tests.

\begin{figure}[t]
    \centering
    \begin{adjustbox}{width=\textwidth}
        \begin{tikzpicture}
    \begin{groupplot}[%
        group style={group size=2 by 2, vertical sep=2.25cm, horizontal sep=2cm},
        grid=both,
        grid style={line width=.1pt, draw=gray!10},
        major grid style={line width=.2pt,draw=gray!50},
        every axis plot/.append style={very thick},
        %y tick label style={/pgf/number format/sci},
        xtick={1,4,8,12,16,20},
        legend style={draw=none,/tikz/every even column/.append %
        style={column sep=1em},legend columns=-1,at={(1.75,1.35)}}]

        \large

        % Messages sent

        \nextgroupplot[xlabel=$|V| / |A|$, ylabel=DSA-SDP / D-CTS,
        title={(a) Messages sent},legend entries={\textsc{uc\_ndcs},
        \textsc{uc\_agent-based}}]

        \addlegendimage{no markers,mycol1,line width=2.0,loosely dashed}
        \addlegendimage{no markers,mycol2,line width=2.0}

        %% UC_NDCS
        \addplot [dashed,mycol1,line width=1.1] coordinates {%
            (1, 54.41)
            (2, 52.58)
            (3, 48.99)
            (4, 46.32)
            (5, 46.16)
            (6, 44.68)
            (7, 44.58)
            (8, 42.29)
            (9, 41.96)
            (10, 41.59)
            (11, 40.88)
            (12, 39.96)
            (13, 39.65)
            (14, 40.27)
            (15, 40.70)
            (16, 39.45)
            (17, 40.30)
            (18, 39.78)
            (19, 41.77)
            (20, 43.01)
        };

        \addplot [name path=u2, draw=none] plot coordinates {%
            (1, 55.60)
            (2, 52.79)
            (3, 50.06)
            (4, 48.06)
            (5, 46.76)
            (6, 45.80)
            (7, 45.87)
            (8, 43.42)
            (9, 43.21)
            (10, 43.85)
            (11, 42.32)
            (12, 41.46)
            (13, 41.65)
            (14, 41.34)
            (15, 42.86)
            (16, 42.07)
            (17, 44.78)
            (18, 43.75)
            (19, 45.13)
            (20, 47.51)
        };

        \addplot [name path=l2, draw=none] plot coordinates {%
            (1, 53.49)
            (2, 50.99)
            (3, 47.49)
            (4, 45.23)
            (5, 44.50)
            (6, 43.15)
            (7, 42.51)
            (8, 40.51)
            (9, 39.36)
            (10, 39.41)
            (11, 39.31)
            (12, 38.03)
            (13, 38.09)
            (14, 37.73)
            (15, 38.29)
            (16, 37.47)
            (17, 37.66)
            (18, 37.19)
            (19, 37.82)
            (20, 38.35)
        };

        \addplot+ [fill=mycol1!25, opacity=0.6] fill between[of=u2 and l2];

        %% UC_Agent-based
        \addplot [mycol2, line width=1.1] coordinates {%
            (1, 57.56)
            (2, 55.13)
            (3, 51.38)
            (4, 49.13)
            (5, 46.14)
            (6, 44.74)
            (7, 44.30)
            (8, 42.85)
            (9, 42.01)
            (10, 41.54)
            (11, 41.60)
            (12, 41.68)
            (13, 41.41)
            (14, 40.77)
            (15, 39.88)
            (16, 39.94)
            (17, 40.03)
            (18, 40.92)
            (19, 40.67)
            (20, 41.26)
        };

        \addplot [name path=u1, draw=none] plot coordinates {%
            (1, 58.62)
            (2, 54.17)
            (3, 49.91)
            (4, 47.57)
            (5, 44.18)
            (6, 43.37)
            (7, 42.62)
            (8, 41.71)
            (9, 40.98)
            (10, 40.74)
            (11, 40.65)
            (12, 40.26)
            (13, 40.63)
            (14, 40.20)
            (15, 39.18)
            (16, 39.08)
            (17, 38.63)
            (18, 38.75)
            (19, 38.51)
            (20, 39.87)
        };

        \addplot [name path=l1, draw=none] plot coordinates {%
            (1, 55.80)
            (2, 55.50)
            (3, 52.39)
            (4, 49.86)
            (5, 47.59)
            (6, 45.37)
            (7, 45.98)
            (8, 43.23)
            (9, 42.87)
            (10, 42.85)
            (11, 42.54)
            (12, 42.62)
            (13, 43.37)
            (14, 41.45)
            (15, 41.71)
            (16, 41.64)
            (17, 41.30)
            (18, 41.73)
            (19, 43.94)
            (20, 45.34)
        };

        \addplot+ [fill=mycol2!25,opacity=0.6,postaction={pattern color = mycol2!40, pattern=north east lines}] fill between[of=u1 and l1];

        % Network load
        \nextgroupplot[xlabel=$|V| / |A|$, ylabel=DSA-SDP / D-CTS,
        title={(b) Network load}]

        %% UC_NDCS
        \addplot [dashed,mycol1,line width=1.1] coordinates {%
            (1, 0.88)
            (2, 0.79)
            (3, 0.74)
            (4, 0.69)
            (5, 0.66)
            (6, 0.64)
            (7, 0.64)
            (8, 0.60)
            (9, 0.59)
            (10, 0.58)
            (11, 0.58)
            (12, 0.56)
            (13, 0.56)
            (14, 0.53)
            (15, 0.57)
            (16, 0.55)
            (17, 0.56)
            (18, 0.56)
            (19, 0.57)
            (20, 0.60)
        };

        \addplot [name path=u2, draw=none] plot coordinates {%
            (1, 0.92)
            (2, 0.79)
            (3, 0.74)
            (4, 0.70)
            (5, 0.69)
            (6, 0.67)
            (7, 0.66)
            (8, 0.62)
            (9, 0.63)
            (10, 0.61)
            (11, 0.61)
            (12, 0.59)
            (13, 0.58)
            (14, 0.57)
            (15, 0.59)
            (16, 0.60)
            (17, 0.60)
            (18, 0.62)
            (19, 0.64)
            (20, 0.66)
        };

        \addplot [name path=l2, draw=none] plot coordinates {%
            (1, 0.88)
            (2, 0.78)
            (3, 0.71)
            (4, 0.67)
            (5, 0.63)
            (6, 0.62)
            (7, 0.61)
            (8, 0.58)
            (9, 0.56)
            (10, 0.55)
            (11, 0.54)
            (12, 0.54)
            (13, 0.53)
            (14, 0.52)
            (15, 0.54)
            (16, 0.52)
            (17, 0.52)
            (18, 0.51)
            (19, 0.52)
            (20, 0.53)
        };

        \addplot+ [fill=mycol1!25, opacity=0.6] fill between[of=u2 and l2];

        %% UC_Agent-based
        \addplot [mycol2,line width=1.1] coordinates {%
            (1, 0.99)
            (2, 0.85)
            (3, 0.76)
            (4, 0.72)
            (5, 0.67)
            (6, 0.63)
            (7, 0.64)
            (8, 0.60)
            (9, 0.59)
            (10, 0.58)
            (11, 0.59)
            (12, 0.57)
            (13, 0.58)
            (14, 0.56)
            (15, 0.55)
            (16, 0.55)
            (17, 0.54)
            (18, 0.55)
            (19, 0.55)
            (20, 0.57)
        };

        \addplot [name path=u1, draw=none] plot coordinates {%
            (1, 0.99)
            (2, 0.82)
            (3, 0.74)
            (4, 0.69)
            (5, 0.63)
            (6, 0.62)
            (7, 0.60)
            (8, 0.59)
            (9, 0.57)
            (10, 0.57)
            (11, 0.57)
            (12, 0.55)
            (13, 0.55)
            (14, 0.55)
            (15, 0.53)
            (16, 0.53)
            (17, 0.53)
            (18, 0.53)
            (19, 0.53)
            (20, 0.54)
        };

        \addplot [name path=l1, draw=none] plot coordinates {%
            (1, 0.99)
            (2, 0.86)
            (3, 0.78)
            (4, 0.73)
            (5, 0.69)
            (6, 0.65)
            (7, 0.66)
            (8, 0.61)
            (9, 0.61)
            (10, 0.60)
            (11, 0.61)
            (12, 0.59)
            (13, 0.60)
            (14, 0.58)
            (15, 0.57)
            (16, 0.57)
            (17, 0.58)
            (18, 0.58)
            (19, 0.59)
            (20, 0.62)
        };

        \addplot+ [fill=mycol2!25,opacity=0.6,postaction={pattern color = mycol2!40, pattern=north east lines}] fill between[of=u1 and l1];

        % NCCCs

        \nextgroupplot[xlabel=$|V| / |A|$, ylabel=DSA-SDP / D-CTS,
        title={(c) NCCCs}]

        %% UC_NDCS
        \addplot [dashed,mycol1,line width=1.1] coordinates {%
            (1, 99.97)
            (2, 91.34)
            (3, 83.46)
            (4, 79.80)
            (5, 79.59)
            (6, 74.60)
            (7, 70.52)
            (8, 67.96)
            (9, 66.43)
            (10, 63.96)
            (11, 69.10)
            (12, 70.82)
            (13, 64.34)
            (14, 56.91)
            (15, 62.32)
            (16, 55.12)
            (17, 60.25)
            (18, 56.48)
            (19, 60.88)
            (20, 64.98)
        };

        \addplot [name path=u2, draw=none] plot coordinates {%
            (1, 100.78)
            (2, 94.72)
            (3, 88.09)
            (4, 83.16)
            (5, 85.67)
            (6, 81.82)
            (7, 74.18)
            (8, 72.85)
            (9, 72.04)
            (10, 70.45)
            (11, 80.55)
            (12, 80.43)
            (13, 72.45)
            (14, 67.39)
            (15, 71.75)
            (16, 62.41)
            (17, 73.47)
            (18, 66.66)
            (19, 70.50)
            (20, 71.98)
        };

        \addplot [name path=l2, draw=none] plot coordinates {%
            (1, 98.96)
            (2, 85.65)
            (3, 81.40)
            (4, 77.82)
            (5, 68.54)
            (6, 68.24)
            (7, 63.78)
            (8, 59.46)
            (9, 58.70)
            (10, 59.60)
            (11, 65.20)
            (12, 62.14)
            (13, 55.43)
            (14, 44.99)
            (15, 55.28)
            (16, 47.78)
            (17, 55.58)
            (18, 55.09)
            (19, 57.73)
            (20, 59.34)
        };

        \addplot+ [fill=mycol1!25, opacity=0.6] fill between[of=u2 and l2];

        %% UC_Agent-based
        \addplot [mycol2,line width=1.1] coordinates {%
            (1, 106.27)
            (2, 104.65)
            (3, 93.27)
            (4, 89.44)
            (5, 84.36)
            (6, 85.06)
            (7, 82.49)
            (8, 81.25)
            (9, 77.98)
            (10, 80.19)
            (11, 77.19)
            (12, 72.95)
            (13, 67.72)
            (14, 72.61)
            (15, 69.04)
            (16, 69.13)
            (17, 72.69)
            (18, 71.09)
            (19, 70.29)
            (20, 63.54)
        };

        \addplot [name path=u1, draw=none] plot coordinates {%
            (1, 106.66)
            (2, 107.58)
            (3, 97.56)
            (4, 94.58)
            (5, 88.11)
            (6, 87.21)
            (7, 86.16)
            (8, 82.10)
            (9, 89.49)
            (10, 89.26)
            (11, 81.45)
            (12, 80.58)
            (13, 78.65)
            (14, 85.32)
            (15, 82.78)
            (16, 79.14)
            (17, 86.91)
            (18, 84.66)
            (19, 74.39)
            (20, 80.18)
        };

        \addplot [name path=l1, draw=none] plot coordinates {%
            (1, 105.03)
            (2, 101.43)
            (3, 91.25)
            (4, 85.03)
            (5, 81.25)
            (6, 80.50)
            (7, 75.37)
            (8, 75.06)
            (9, 67.18)
            (10, 73.13)
            (11, 67.03)
            (12, 64.18)
            (13, 64.06)
            (14, 67.36)
            (15, 61.67)
            (16, 64.66)
            (17, 68.79)
            (18, 63.25)
            (19, 63.53)
            (20, 58.40)
        };

        \addplot+ [fill=mycol2!25,opacity=0.6,postaction={pattern color = mycol2!40, pattern=north
        east lines}] fill between[of=u1 and l1];

        % CPU time

        \nextgroupplot[xlabel=$|V| / |A|$, ylabel=DSA-SDP / D-CTS,
        title={(d) CPU time}]

        %% UC_NDCS
        \addplot [dashed,mycol1,line width=1.1] coordinates {%
            (1, 8.50)
            (2, 9.28)
            (3, 9.49)
            (4, 9.25)
            (5, 10.96)
            (6, 11.55)
            (7, 10.68)
            (8, 11.00)
            (9, 11.53)
            (10, 12.47)
            (11, 13.32)
            (12, 12.66)
            (13, 14.65)
            (14, 14.42)
            (15, 15.53)
            (16, 16.17)
            (17, 16.28)
            (18, 18.25)
            (19, 20.93)
            (20, 21.40)
        };

        \addplot [name path=u2, draw=none] plot coordinates {%
            (1, 10.50)
            (2, 10.10)
            (3, 10.91)
            (4, 12.27)
            (5, 12.63)
            (6, 12.53)
            (7, 11.95)
            (8, 11.89)
            (9, 13.24)
            (10, 14.01)
            (11, 15.85)
            (12, 14.53)
            (13, 16.24)
            (14, 15.44)
            (15, 17.21)
            (16, 17.31)
            (17, 19.81)
            (18, 22.63)
            (19, 24.24)
            (20, 24.70)
        };

        \addplot [name path=l2, draw=none] plot coordinates {%
            (1, 7.43)
            (2, 6.70)
            (3, 7.47)
            (4, 7.34)
            (5, 8.96)
            (6, 9.61)
            (7, 9.12)
            (8, 9.00)
            (9, 9.97)
            (10, 11.02)
            (11, 11.88)
            (12, 11.44)
            (13, 12.26)
            (14, 13.00)
            (15, 13.31)
            (16, 12.96)
            (17, 14.58)
            (18, 15.18)
            (19, 16.81)
            (20, 17.02)
        };

        \addplot+ [fill=mycol1!25, opacity=0.6] fill between[of=u2 and l2];

        %% UC_Agent-based
        \addplot [mycol2,line width=1.1] coordinates {%
            (1, 8.49)
            (2, 10.52)
            (3, 11.96)
            (4, 12.38)
            (5, 11.22)
            (6, 11.73)
            (7, 11.56)
            (8, 12.24)
            (9, 13.15)
            (10, 14.31)
            (11, 14.44)
            (12, 14.95)
            (13, 15.96)
            (14, 16.23)
            (15, 16.19)
            (16, 17.92)
            (17, 18.63)
            (18, 19.48)
            (19, 20.96)
            (20, 21.74)
        };

        \addplot [name path=u1, draw=none] plot coordinates {%
            (1, 9.00)
            (2, 12.18)
            (3, 12.73)
            (4, 13.93)
            (5, 14.40)
            (6, 13.68)
            (7, 13.51)
            (8, 13.52)
            (9, 14.14)
            (10, 15.52)
            (11, 15.60)
            (12, 17.02)
            (13, 17.41)
            (14, 17.46)
            (15, 18.68)
            (16, 19.62)
            (17, 20.80)
            (18, 24.15)
            (19, 23.25)
            (20, 27.50)
        };

        \addplot [name path=l1, draw=none] plot coordinates {%
            (1, 7.31)
            (2, 9.38)
            (3, 9.74)
            (4, 9.76)
            (5, 9.86)
            (6, 10.52)
            (7, 10.59)
            (8, 11.01)
            (9, 11.35)
            (10, 12.45)
            (11, 12.25)
            (12, 13.58)
            (13, 13.70)
            (14, 14.51)
            (15, 14.61)
            (16, 15.72)
            (17, 16.49)
            (18, 17.40)
            (19, 18.05)
            (20, 18.34)
        };

        \addplot+ [fill=mycol2!25,opacity=0.6,postaction={pattern color = mycol2!40, pattern=north east lines}] fill between[of=u1 and l1];
    \end{groupplot}
\end{tikzpicture}

    \end{adjustbox}
    \caption[Ratio of DSA-SDP performance to D-CTS performance]{%
        Ratio of DSA-SDP performance to D-CTS performance. Each subfigure denotes a
        performance metric $m$, while each point is the median and $95\%$ confidence
        interval over $100$ problems of $m_a / m_b$, where $m_a$ (resp. $m_b$) is the
        value of DSA-SDP (resp. D-CTS) for $m$. The X-axis is the task-to-agent ratio.}
    \label{fig:dcts-t2}
\end{figure}

The remaining metrics put DSA-SDP at a distinct disadvantage (Figure \ref{fig:dcts-t2}a,
c, d). The overload compared to D-CTS is $41.72 \pm [12.45, 0.42]$ times more messages
sent, $72.78 \pm [34.79, 27.79]$ times more NCCCs, and $13.82 \pm [4.52, 3.71]$ times more
CPU time. This is explained as follows. While the number of messages sent is $O(|V| \cdot
|A|)$ in D-CTS (Section \ref{sec:dcts}), it is $O(|V| \cdot |A|^2)$ in DSA-SDP, since the
agents exchange their assignments\footnote{To align with Table \ref{t:dcop_chars}, we have
that $t = |V|$, and $n = l = |A|$.} \cite{zivan2014}. In D-CTS, analysing in sequence the
agents that can be assigned to each task (line $4$ in Algorithm \ref{algo:d2}) requires
$O(|V| \cdot |A|)$ NCCCs. DSA-SDP does a similar analysis, but for each message exchanged
between two agents, which requires $O(|V|^2 \cdot |A|^2)$ NCCCs. Finally, the time
complexity of DSA-SDP is $O(\tmax \cdot |V| \cdot |A|^2)$, where $O(|V| \cdot |A|)$ is
required by the message exchange phase at each time, and $O(|A|)$ is required by each
agent to calculate the assignment costs (Equation \ref{eq:costs}). Hence, DSA-SDP is
asymptotically slower than D-CTS (Equations \ref{eq:comp1} and \ref{eq:comp2}). Overall,
D-CTS took $525 \pm [281, 482]$ ms, while DSA-SDP took $6.97 \pm [5.84, 6.2]$ seconds. In
accordance with the above, the ratio of DSA-SDP performance to D-CTS performance tends to
increase with regard to CPU time, and to decrease with regard to the other metrics.

In a dynamic environment, desirable features of a distributed algorithm include being
robust to disruptions and minimising communication overhead (Section \ref{sec:dcts}).
The latter feature is particularly important in real-world domains such as disaster
response, where agent communication can be costly (i.e., not free-comm environments,
Section \ref{sec:mdp}) or there may be operational constraints, such as low bandwidth or
limited network topology (e.g., sparse robot swarms searching for shipwrecks on the
seabed, or monitoring forest fires \cite{tarapore2020}). In our tests, compared to the
state-of-the-art DSA-SDP, D-CTS achieves a slightly better solution quality (Figure
\ref{fig:dcts-t}), and is one order of magnitude more efficient in terms of communication
overhead and time complexity (Figure \ref{fig:dcts-t2}). This affirms its effectiveness as
a scalable and distributed CFSTP algorithm for dynamic environments.

\section{Summary}

We proposed a novel and minimal BIP of the CFSTP, and demonstrated its equivalence with
the original MIP formulation (Section \ref{sec:mip}). Based on a novel reduction of the
CFSTP to a DynDCOP, we also created D-CTS, a distributed version of CTS that preserves its
properties (anytime, convergent, scalable) and is self-stabilising. We defined a
large-scale CFSTP dataset, as well as a dynamic test framework, and empirically showed
that D-CTS has slightly better performance than the state-of-the-art DSA-SDP, while having
significantly lower communication overhead and time complexity. Having filled the main
gaps in the literature, in order to meet all our research objectives (Section
\ref{sec:objectives}), the next and final chapter will focus on the main limitations of
the CFSTP model itself.
