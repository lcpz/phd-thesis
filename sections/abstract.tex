\begin{abstract}

The \emph{Coalition Formation with Spatial and Temporal constraints Problem} (CFSTP) is
designed to characterise scenarios at the intersection between task allocation and
coalition formation. In this model, tens of heterogeneous agents are deployed over
kilometre-wide areas to carry out thousands of tasks, each with its deadline and workload.
To maximise the number of tasks completed, the agents need to cooperate by forming,
disbanding and reforming coalitions.

In this thesis, we start with an in-depth analysis of \emph{Coalition Formation with
Look-Ahead} (CFLA), the state-of-the-art CFSTP algorithm. We outline its main limitations,
based on which we derive an extension called CFLA2. We show that we cannot eliminate the
limitations of CFLA in CFLA2, hence we also develop a novel algorithm called
\emph{Cluster-based Task Scheduling} (CTS), which is the first to be simultaneously
anytime, efficient and with convergence guarantee. We empirically demonstrate the
superiority of CTS over CFLA and CFLA2, and propose S-CTS, a simplified but parallel and
more efficient variant. In problems generated by the RoboCup Rescue Simulation, S-CTS can
compete with the high-performance Binary Max-Sum and DSA algorithms, while being up to two
orders of magnitude faster.

We then propose a minimal mathematical program of the CFSTP, and reduce it to a Dynamic
and Distributed Constraint Optimisation Problem, based on which we design D-CTS, a
distributed version of CTS. We create a test framework that simulates the mobilisation of
firefighters, which we use to show the effectiveness of D-CTS in large-scale and dynamic
environments.

Finally, to characterise scenarios in which the faster the tasks are solved, the greater
the benefits, we propose the \emph{Multi-Agent Routing and Scheduling through Coalition
Formation problem} (MARSC), a generalisation of both the CFSTP and the important Team
Orienteering Problem with Time Windows. We formulate a binary integer program and propose
\emph{Anytime, exact and parallel Node Traversal} (ANT), the first algorithm of its kind
for both the MARSC and the CFSTP. Moreover, we define an approximate variant called
ANT-$\varepsilon$. Both algorithms are validated in our realistic test framework, using as
baselines an extended version of CTS, and an implementation of the Earliest Deadline First
technique, which is typically used in real-time systems.

\iffalse
\begin{flushleft}
Keywords:
Multi-Agent Task Allocation \( \cdot \)
Routing \( \cdot \)
Scheduling \( \cdot \)
Dynamic and Distributed Constraint Optimisation \( \cdot \)
Real-Time Systems \( \cdot \)
Large-Scale Disaster Response
\end{flushleft}
\fi

\end{abstract}
